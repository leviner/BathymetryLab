
% Default to the notebook output style

    


% Inherit from the specified cell style.




    
\documentclass[11pt]{article}

    
    
    \usepackage[T1]{fontenc}
    % Nicer default font (+ math font) than Computer Modern for most use cases
    \usepackage{mathpazo}

    % Basic figure setup, for now with no caption control since it's done
    % automatically by Pandoc (which extracts ![](path) syntax from Markdown).
    \usepackage{graphicx}
    % We will generate all images so they have a width \maxwidth. This means
    % that they will get their normal width if they fit onto the page, but
    % are scaled down if they would overflow the margins.
    \makeatletter
    \def\maxwidth{\ifdim\Gin@nat@width>\linewidth\linewidth
    \else\Gin@nat@width\fi}
    \makeatother
    \let\Oldincludegraphics\includegraphics
    % Set max figure width to be 80% of text width, for now hardcoded.
    \renewcommand{\includegraphics}[1]{\Oldincludegraphics[width=.8\maxwidth]{#1}}
    % Ensure that by default, figures have no caption (until we provide a
    % proper Figure object with a Caption API and a way to capture that
    % in the conversion process - todo).
    \usepackage{caption}
    \DeclareCaptionLabelFormat{nolabel}{}
    \captionsetup{labelformat=nolabel}

    \usepackage{adjustbox} % Used to constrain images to a maximum size 
    \usepackage{xcolor} % Allow colors to be defined
    \usepackage{enumerate} % Needed for markdown enumerations to work
    \usepackage{geometry} % Used to adjust the document margins
    \usepackage{amsmath} % Equations
    \usepackage{amssymb} % Equations
    \usepackage{textcomp} % defines textquotesingle
    % Hack from http://tex.stackexchange.com/a/47451/13684:
    \AtBeginDocument{%
        \def\PYZsq{\textquotesingle}% Upright quotes in Pygmentized code
    }
    \usepackage{upquote} % Upright quotes for verbatim code
    \usepackage{eurosym} % defines \euro
    \usepackage[mathletters]{ucs} % Extended unicode (utf-8) support
    \usepackage[utf8x]{inputenc} % Allow utf-8 characters in the tex document
    \usepackage{fancyvrb} % verbatim replacement that allows latex
    \usepackage{grffile} % extends the file name processing of package graphics 
                         % to support a larger range 
    % The hyperref package gives us a pdf with properly built
    % internal navigation ('pdf bookmarks' for the table of contents,
    % internal cross-reference links, web links for URLs, etc.)
    \usepackage{hyperref}
    \usepackage{longtable} % longtable support required by pandoc >1.10
    \usepackage{booktabs}  % table support for pandoc > 1.12.2
    \usepackage[inline]{enumitem} % IRkernel/repr support (it uses the enumerate* environment)
    \usepackage[normalem]{ulem} % ulem is needed to support strikethroughs (\sout)
                                % normalem makes italics be italics, not underlines
    

    
    
    % Colors for the hyperref package
    \definecolor{urlcolor}{rgb}{0,.145,.698}
    \definecolor{linkcolor}{rgb}{.71,0.21,0.01}
    \definecolor{citecolor}{rgb}{.12,.54,.11}

    % ANSI colors
    \definecolor{ansi-black}{HTML}{3E424D}
    \definecolor{ansi-black-intense}{HTML}{282C36}
    \definecolor{ansi-red}{HTML}{E75C58}
    \definecolor{ansi-red-intense}{HTML}{B22B31}
    \definecolor{ansi-green}{HTML}{00A250}
    \definecolor{ansi-green-intense}{HTML}{007427}
    \definecolor{ansi-yellow}{HTML}{DDB62B}
    \definecolor{ansi-yellow-intense}{HTML}{B27D12}
    \definecolor{ansi-blue}{HTML}{208FFB}
    \definecolor{ansi-blue-intense}{HTML}{0065CA}
    \definecolor{ansi-magenta}{HTML}{D160C4}
    \definecolor{ansi-magenta-intense}{HTML}{A03196}
    \definecolor{ansi-cyan}{HTML}{60C6C8}
    \definecolor{ansi-cyan-intense}{HTML}{258F8F}
    \definecolor{ansi-white}{HTML}{C5C1B4}
    \definecolor{ansi-white-intense}{HTML}{A1A6B2}

    % commands and environments needed by pandoc snippets
    % extracted from the output of `pandoc -s`
    \providecommand{\tightlist}{%
      \setlength{\itemsep}{0pt}\setlength{\parskip}{0pt}}
    \DefineVerbatimEnvironment{Highlighting}{Verbatim}{commandchars=\\\{\}}
    % Add ',fontsize=\small' for more characters per line
    \newenvironment{Shaded}{}{}
    \newcommand{\KeywordTok}[1]{\textcolor[rgb]{0.00,0.44,0.13}{\textbf{{#1}}}}
    \newcommand{\DataTypeTok}[1]{\textcolor[rgb]{0.56,0.13,0.00}{{#1}}}
    \newcommand{\DecValTok}[1]{\textcolor[rgb]{0.25,0.63,0.44}{{#1}}}
    \newcommand{\BaseNTok}[1]{\textcolor[rgb]{0.25,0.63,0.44}{{#1}}}
    \newcommand{\FloatTok}[1]{\textcolor[rgb]{0.25,0.63,0.44}{{#1}}}
    \newcommand{\CharTok}[1]{\textcolor[rgb]{0.25,0.44,0.63}{{#1}}}
    \newcommand{\StringTok}[1]{\textcolor[rgb]{0.25,0.44,0.63}{{#1}}}
    \newcommand{\CommentTok}[1]{\textcolor[rgb]{0.38,0.63,0.69}{\textit{{#1}}}}
    \newcommand{\OtherTok}[1]{\textcolor[rgb]{0.00,0.44,0.13}{{#1}}}
    \newcommand{\AlertTok}[1]{\textcolor[rgb]{1.00,0.00,0.00}{\textbf{{#1}}}}
    \newcommand{\FunctionTok}[1]{\textcolor[rgb]{0.02,0.16,0.49}{{#1}}}
    \newcommand{\RegionMarkerTok}[1]{{#1}}
    \newcommand{\ErrorTok}[1]{\textcolor[rgb]{1.00,0.00,0.00}{\textbf{{#1}}}}
    \newcommand{\NormalTok}[1]{{#1}}
    
    % Additional commands for more recent versions of Pandoc
    \newcommand{\ConstantTok}[1]{\textcolor[rgb]{0.53,0.00,0.00}{{#1}}}
    \newcommand{\SpecialCharTok}[1]{\textcolor[rgb]{0.25,0.44,0.63}{{#1}}}
    \newcommand{\VerbatimStringTok}[1]{\textcolor[rgb]{0.25,0.44,0.63}{{#1}}}
    \newcommand{\SpecialStringTok}[1]{\textcolor[rgb]{0.73,0.40,0.53}{{#1}}}
    \newcommand{\ImportTok}[1]{{#1}}
    \newcommand{\DocumentationTok}[1]{\textcolor[rgb]{0.73,0.13,0.13}{\textit{{#1}}}}
    \newcommand{\AnnotationTok}[1]{\textcolor[rgb]{0.38,0.63,0.69}{\textbf{\textit{{#1}}}}}
    \newcommand{\CommentVarTok}[1]{\textcolor[rgb]{0.38,0.63,0.69}{\textbf{\textit{{#1}}}}}
    \newcommand{\VariableTok}[1]{\textcolor[rgb]{0.10,0.09,0.49}{{#1}}}
    \newcommand{\ControlFlowTok}[1]{\textcolor[rgb]{0.00,0.44,0.13}{\textbf{{#1}}}}
    \newcommand{\OperatorTok}[1]{\textcolor[rgb]{0.40,0.40,0.40}{{#1}}}
    \newcommand{\BuiltInTok}[1]{{#1}}
    \newcommand{\ExtensionTok}[1]{{#1}}
    \newcommand{\PreprocessorTok}[1]{\textcolor[rgb]{0.74,0.48,0.00}{{#1}}}
    \newcommand{\AttributeTok}[1]{\textcolor[rgb]{0.49,0.56,0.16}{{#1}}}
    \newcommand{\InformationTok}[1]{\textcolor[rgb]{0.38,0.63,0.69}{\textbf{\textit{{#1}}}}}
    \newcommand{\WarningTok}[1]{\textcolor[rgb]{0.38,0.63,0.69}{\textbf{\textit{{#1}}}}}
    
    
    % Define a nice break command that doesn't care if a line doesn't already
    % exist.
    \def\br{\hspace*{\fill} \\* }
    % Math Jax compatability definitions
    \def\gt{>}
    \def\lt{<}
    % Document parameters
    \title{AcousticSensorLab-ForPublish}
    
    
    

    % Pygments definitions
    
\makeatletter
\def\PY@reset{\let\PY@it=\relax \let\PY@bf=\relax%
    \let\PY@ul=\relax \let\PY@tc=\relax%
    \let\PY@bc=\relax \let\PY@ff=\relax}
\def\PY@tok#1{\csname PY@tok@#1\endcsname}
\def\PY@toks#1+{\ifx\relax#1\empty\else%
    \PY@tok{#1}\expandafter\PY@toks\fi}
\def\PY@do#1{\PY@bc{\PY@tc{\PY@ul{%
    \PY@it{\PY@bf{\PY@ff{#1}}}}}}}
\def\PY#1#2{\PY@reset\PY@toks#1+\relax+\PY@do{#2}}

\expandafter\def\csname PY@tok@w\endcsname{\def\PY@tc##1{\textcolor[rgb]{0.73,0.73,0.73}{##1}}}
\expandafter\def\csname PY@tok@c\endcsname{\let\PY@it=\textit\def\PY@tc##1{\textcolor[rgb]{0.25,0.50,0.50}{##1}}}
\expandafter\def\csname PY@tok@cp\endcsname{\def\PY@tc##1{\textcolor[rgb]{0.74,0.48,0.00}{##1}}}
\expandafter\def\csname PY@tok@k\endcsname{\let\PY@bf=\textbf\def\PY@tc##1{\textcolor[rgb]{0.00,0.50,0.00}{##1}}}
\expandafter\def\csname PY@tok@kp\endcsname{\def\PY@tc##1{\textcolor[rgb]{0.00,0.50,0.00}{##1}}}
\expandafter\def\csname PY@tok@kt\endcsname{\def\PY@tc##1{\textcolor[rgb]{0.69,0.00,0.25}{##1}}}
\expandafter\def\csname PY@tok@o\endcsname{\def\PY@tc##1{\textcolor[rgb]{0.40,0.40,0.40}{##1}}}
\expandafter\def\csname PY@tok@ow\endcsname{\let\PY@bf=\textbf\def\PY@tc##1{\textcolor[rgb]{0.67,0.13,1.00}{##1}}}
\expandafter\def\csname PY@tok@nb\endcsname{\def\PY@tc##1{\textcolor[rgb]{0.00,0.50,0.00}{##1}}}
\expandafter\def\csname PY@tok@nf\endcsname{\def\PY@tc##1{\textcolor[rgb]{0.00,0.00,1.00}{##1}}}
\expandafter\def\csname PY@tok@nc\endcsname{\let\PY@bf=\textbf\def\PY@tc##1{\textcolor[rgb]{0.00,0.00,1.00}{##1}}}
\expandafter\def\csname PY@tok@nn\endcsname{\let\PY@bf=\textbf\def\PY@tc##1{\textcolor[rgb]{0.00,0.00,1.00}{##1}}}
\expandafter\def\csname PY@tok@ne\endcsname{\let\PY@bf=\textbf\def\PY@tc##1{\textcolor[rgb]{0.82,0.25,0.23}{##1}}}
\expandafter\def\csname PY@tok@nv\endcsname{\def\PY@tc##1{\textcolor[rgb]{0.10,0.09,0.49}{##1}}}
\expandafter\def\csname PY@tok@no\endcsname{\def\PY@tc##1{\textcolor[rgb]{0.53,0.00,0.00}{##1}}}
\expandafter\def\csname PY@tok@nl\endcsname{\def\PY@tc##1{\textcolor[rgb]{0.63,0.63,0.00}{##1}}}
\expandafter\def\csname PY@tok@ni\endcsname{\let\PY@bf=\textbf\def\PY@tc##1{\textcolor[rgb]{0.60,0.60,0.60}{##1}}}
\expandafter\def\csname PY@tok@na\endcsname{\def\PY@tc##1{\textcolor[rgb]{0.49,0.56,0.16}{##1}}}
\expandafter\def\csname PY@tok@nt\endcsname{\let\PY@bf=\textbf\def\PY@tc##1{\textcolor[rgb]{0.00,0.50,0.00}{##1}}}
\expandafter\def\csname PY@tok@nd\endcsname{\def\PY@tc##1{\textcolor[rgb]{0.67,0.13,1.00}{##1}}}
\expandafter\def\csname PY@tok@s\endcsname{\def\PY@tc##1{\textcolor[rgb]{0.73,0.13,0.13}{##1}}}
\expandafter\def\csname PY@tok@sd\endcsname{\let\PY@it=\textit\def\PY@tc##1{\textcolor[rgb]{0.73,0.13,0.13}{##1}}}
\expandafter\def\csname PY@tok@si\endcsname{\let\PY@bf=\textbf\def\PY@tc##1{\textcolor[rgb]{0.73,0.40,0.53}{##1}}}
\expandafter\def\csname PY@tok@se\endcsname{\let\PY@bf=\textbf\def\PY@tc##1{\textcolor[rgb]{0.73,0.40,0.13}{##1}}}
\expandafter\def\csname PY@tok@sr\endcsname{\def\PY@tc##1{\textcolor[rgb]{0.73,0.40,0.53}{##1}}}
\expandafter\def\csname PY@tok@ss\endcsname{\def\PY@tc##1{\textcolor[rgb]{0.10,0.09,0.49}{##1}}}
\expandafter\def\csname PY@tok@sx\endcsname{\def\PY@tc##1{\textcolor[rgb]{0.00,0.50,0.00}{##1}}}
\expandafter\def\csname PY@tok@m\endcsname{\def\PY@tc##1{\textcolor[rgb]{0.40,0.40,0.40}{##1}}}
\expandafter\def\csname PY@tok@gh\endcsname{\let\PY@bf=\textbf\def\PY@tc##1{\textcolor[rgb]{0.00,0.00,0.50}{##1}}}
\expandafter\def\csname PY@tok@gu\endcsname{\let\PY@bf=\textbf\def\PY@tc##1{\textcolor[rgb]{0.50,0.00,0.50}{##1}}}
\expandafter\def\csname PY@tok@gd\endcsname{\def\PY@tc##1{\textcolor[rgb]{0.63,0.00,0.00}{##1}}}
\expandafter\def\csname PY@tok@gi\endcsname{\def\PY@tc##1{\textcolor[rgb]{0.00,0.63,0.00}{##1}}}
\expandafter\def\csname PY@tok@gr\endcsname{\def\PY@tc##1{\textcolor[rgb]{1.00,0.00,0.00}{##1}}}
\expandafter\def\csname PY@tok@ge\endcsname{\let\PY@it=\textit}
\expandafter\def\csname PY@tok@gs\endcsname{\let\PY@bf=\textbf}
\expandafter\def\csname PY@tok@gp\endcsname{\let\PY@bf=\textbf\def\PY@tc##1{\textcolor[rgb]{0.00,0.00,0.50}{##1}}}
\expandafter\def\csname PY@tok@go\endcsname{\def\PY@tc##1{\textcolor[rgb]{0.53,0.53,0.53}{##1}}}
\expandafter\def\csname PY@tok@gt\endcsname{\def\PY@tc##1{\textcolor[rgb]{0.00,0.27,0.87}{##1}}}
\expandafter\def\csname PY@tok@err\endcsname{\def\PY@bc##1{\setlength{\fboxsep}{0pt}\fcolorbox[rgb]{1.00,0.00,0.00}{1,1,1}{\strut ##1}}}
\expandafter\def\csname PY@tok@kc\endcsname{\let\PY@bf=\textbf\def\PY@tc##1{\textcolor[rgb]{0.00,0.50,0.00}{##1}}}
\expandafter\def\csname PY@tok@kd\endcsname{\let\PY@bf=\textbf\def\PY@tc##1{\textcolor[rgb]{0.00,0.50,0.00}{##1}}}
\expandafter\def\csname PY@tok@kn\endcsname{\let\PY@bf=\textbf\def\PY@tc##1{\textcolor[rgb]{0.00,0.50,0.00}{##1}}}
\expandafter\def\csname PY@tok@kr\endcsname{\let\PY@bf=\textbf\def\PY@tc##1{\textcolor[rgb]{0.00,0.50,0.00}{##1}}}
\expandafter\def\csname PY@tok@bp\endcsname{\def\PY@tc##1{\textcolor[rgb]{0.00,0.50,0.00}{##1}}}
\expandafter\def\csname PY@tok@fm\endcsname{\def\PY@tc##1{\textcolor[rgb]{0.00,0.00,1.00}{##1}}}
\expandafter\def\csname PY@tok@vc\endcsname{\def\PY@tc##1{\textcolor[rgb]{0.10,0.09,0.49}{##1}}}
\expandafter\def\csname PY@tok@vg\endcsname{\def\PY@tc##1{\textcolor[rgb]{0.10,0.09,0.49}{##1}}}
\expandafter\def\csname PY@tok@vi\endcsname{\def\PY@tc##1{\textcolor[rgb]{0.10,0.09,0.49}{##1}}}
\expandafter\def\csname PY@tok@vm\endcsname{\def\PY@tc##1{\textcolor[rgb]{0.10,0.09,0.49}{##1}}}
\expandafter\def\csname PY@tok@sa\endcsname{\def\PY@tc##1{\textcolor[rgb]{0.73,0.13,0.13}{##1}}}
\expandafter\def\csname PY@tok@sb\endcsname{\def\PY@tc##1{\textcolor[rgb]{0.73,0.13,0.13}{##1}}}
\expandafter\def\csname PY@tok@sc\endcsname{\def\PY@tc##1{\textcolor[rgb]{0.73,0.13,0.13}{##1}}}
\expandafter\def\csname PY@tok@dl\endcsname{\def\PY@tc##1{\textcolor[rgb]{0.73,0.13,0.13}{##1}}}
\expandafter\def\csname PY@tok@s2\endcsname{\def\PY@tc##1{\textcolor[rgb]{0.73,0.13,0.13}{##1}}}
\expandafter\def\csname PY@tok@sh\endcsname{\def\PY@tc##1{\textcolor[rgb]{0.73,0.13,0.13}{##1}}}
\expandafter\def\csname PY@tok@s1\endcsname{\def\PY@tc##1{\textcolor[rgb]{0.73,0.13,0.13}{##1}}}
\expandafter\def\csname PY@tok@mb\endcsname{\def\PY@tc##1{\textcolor[rgb]{0.40,0.40,0.40}{##1}}}
\expandafter\def\csname PY@tok@mf\endcsname{\def\PY@tc##1{\textcolor[rgb]{0.40,0.40,0.40}{##1}}}
\expandafter\def\csname PY@tok@mh\endcsname{\def\PY@tc##1{\textcolor[rgb]{0.40,0.40,0.40}{##1}}}
\expandafter\def\csname PY@tok@mi\endcsname{\def\PY@tc##1{\textcolor[rgb]{0.40,0.40,0.40}{##1}}}
\expandafter\def\csname PY@tok@il\endcsname{\def\PY@tc##1{\textcolor[rgb]{0.40,0.40,0.40}{##1}}}
\expandafter\def\csname PY@tok@mo\endcsname{\def\PY@tc##1{\textcolor[rgb]{0.40,0.40,0.40}{##1}}}
\expandafter\def\csname PY@tok@ch\endcsname{\let\PY@it=\textit\def\PY@tc##1{\textcolor[rgb]{0.25,0.50,0.50}{##1}}}
\expandafter\def\csname PY@tok@cm\endcsname{\let\PY@it=\textit\def\PY@tc##1{\textcolor[rgb]{0.25,0.50,0.50}{##1}}}
\expandafter\def\csname PY@tok@cpf\endcsname{\let\PY@it=\textit\def\PY@tc##1{\textcolor[rgb]{0.25,0.50,0.50}{##1}}}
\expandafter\def\csname PY@tok@c1\endcsname{\let\PY@it=\textit\def\PY@tc##1{\textcolor[rgb]{0.25,0.50,0.50}{##1}}}
\expandafter\def\csname PY@tok@cs\endcsname{\let\PY@it=\textit\def\PY@tc##1{\textcolor[rgb]{0.25,0.50,0.50}{##1}}}

\def\PYZbs{\char`\\}
\def\PYZus{\char`\_}
\def\PYZob{\char`\{}
\def\PYZcb{\char`\}}
\def\PYZca{\char`\^}
\def\PYZam{\char`\&}
\def\PYZlt{\char`\<}
\def\PYZgt{\char`\>}
\def\PYZsh{\char`\#}
\def\PYZpc{\char`\%}
\def\PYZdl{\char`\$}
\def\PYZhy{\char`\-}
\def\PYZsq{\char`\'}
\def\PYZdq{\char`\"}
\def\PYZti{\char`\~}
% for compatibility with earlier versions
\def\PYZat{@}
\def\PYZlb{[}
\def\PYZrb{]}
\makeatother


    % Exact colors from NB
    \definecolor{incolor}{rgb}{0.0, 0.0, 0.5}
    \definecolor{outcolor}{rgb}{0.545, 0.0, 0.0}



    
    % Prevent overflowing lines due to hard-to-break entities
    \sloppy 
    % Setup hyperref package
    \hypersetup{
      breaklinks=true,  % so long urls are correctly broken across lines
      colorlinks=true,
      urlcolor=urlcolor,
      linkcolor=linkcolor,
      citecolor=citecolor,
      }
    % Slightly bigger margins than the latex defaults
    
    \geometry{verbose,tmargin=1in,bmargin=1in,lmargin=1in,rmargin=1in}
    
    

    \begin{document}
    
    
    \maketitle
    
    

    
    Laboratory Excercise: Acoustic Sensors

\hypertarget{milestone-items}{%
\subsection{Milestone Items}\label{milestone-items}}

The following are required to complete the milestone for this excersize:
- Part 1 1. Publish your estimated value of the speed of sound in air to
the class MQTT server 2. Submit to Canvas all three figures you generate
using your three vectors of \texttt{c} values 3. Submit to Canvas the
python code used to create the three figures - Part 2 1. Submit to
Canvas your figure of the bathymetry along the dock 2. Submit to Canvas
the python code used to create your figure Provide a 2-3 sentence
response addressing the questions proposed on the milestone.

\hypertarget{part-1-determining-the-speed-of-sound-in-air}{%
\section{Part 1: Determining the Speed of Sound in
Air}\label{part-1-determining-the-speed-of-sound-in-air}}

For this activity, you will be using a \textbf{HCSR04} 40 kHz ultrasonic
sensor to determine the speed of sound in air. This sensor sends out a
pulse of sound through one transducer, and listens for the echo return
in the second transducer. Based on the delay between sending the signal
and receiving an echo back, you will be able to determine how far away
an object is from the sensor.\\
In this instance we are unsure of the speed of sound in air. By
collecting measurements at a known distance, we can see how changing the
speed of sound affects the accuracy of distance the sensor calculates,
and narrow in on the correct value.

\hypertarget{assembling-your-hcsr04-sensor}{%
\subsection{Assembling your HCSR04
Sensor}\label{assembling-your-hcsr04-sensor}}

Begin by connecting your \textbf{HCSR04} to your \textbf{ESP8266}. The
HCSR04 requires 5 volts, so for this excersize you will use your
microcontroller in a breadboard, powered by the USB cable from your
computer. The USB cable provides 5V to the board, which means we can use
the \emph{V+} pin on the ESP8266 to give 5V to the HCSR04.

The HCSR04 sensor has 4 pins, \emph{GND}, \emph{VCC}, \emph{Trig}, and
\emph{Echo}. For this excercise, connect the \emph{trig} pin on the
HCSR04 to GPIO pin 12 on the ESP8266, and the \emph{echo} pin to GPIO
pin 14. Connect \emph{GND} to \emph{GND} and \emph{VCC} to \emph{V+}.

The driver for this sensor is already included in the firmware on your
\textbf{ESP8266} as a module named \texttt{hcsr04}. We can define our
sensor using the following code. We will begin by estimating that the
speed of sound in air is \textbf{300 meters per second}, so we will
define our variable \texttt{c} as 300:

\texttt{import\ hcsr04\ sensor\ =\ hcsr04.HCSR04(trigger\_pin\ =\ 12,\ echo\_pin\ =\ 14,\ c\ =\ 300)}

The \texttt{hcsr04.HCSR04} class requires three input variable: -
\texttt{trigger\_pin}, the GPIO pin on the ESP8266 corresponding to the
\textbf{trig} pin on the HCSR04 - \texttt{echo\_pin}, the GPIO pin on
the ESP8266 corresponding to the \textbf{echo} pin on the HCSR04 -
\texttt{c}, the speed of sound to use to calculate distance

\hypertarget{measuring-distance}{%
\subsection{Measuring Distance}\label{measuring-distance}}

Once your sensor variable is defined, you can take a measurement by
executing the function \texttt{distance}:

\texttt{sensor.distance()}

Your ESP8266 should return an integer value that corresponds to the
distance of the object in front of the sensor in \textbf{millimeters}.
Try pointing your sensor at an object and moving it closer and further
away from it while taking measurements to see how the value changes.

Test the limits of your sensor. Does there appear to be a minimum
distance you can measure? Does there appear to be a maximum distance?
What happens when it seems like you might be too far away or too close
to take a measurement?

\hypertarget{estimating-c}{%
\subsection{\texorpdfstring{Estimating
\texttt{c}}{Estimating c}}\label{estimating-c}}

We are now going to record a series of measurements at a known distance
while adjusting the value for \texttt{c} to determine the correct value.

First, lets define \texttt{c} as a vector of test values by entering a
new variable on the ESP8266:

\texttt{c\ =\ {[}100,\ 150,\ 200,\ 250,\ 300,\ 350,\ 400,\ 450{]}}

Use the following code to collect distance measurements at each value of
\texttt{c} that you just defined on your ESP8266:

    \begin{Verbatim}[commandchars=\\\{\}]
{\color{incolor}In [{\color{incolor} }]:} \PY{k+kn}{import} \PY{n+nn}{utime} \PY{c+c1}{\PYZsh{} Import the utime library}
        \PY{n}{distance} \PY{o}{=} \PY{p}{[}\PY{p}{]} \PY{c+c1}{\PYZsh{} Define a holding variable for your distance values to be appended }
        \PY{k}{for} \PY{n}{speed} \PY{o+ow}{in} \PY{n}{c}\PY{p}{:} \PY{c+c1}{\PYZsh{} For each value of c... }
            \PY{n}{sensor} \PY{o}{=} \PY{n}{hcsr04}\PY{o}{.}\PY{n}{HCSR04}\PY{p}{(}\PY{n}{trigger\PYZus{}pin} \PY{o}{=} \PY{l+m+mi}{12}\PY{p}{,} \PY{n}{echo\PYZus{}pin} \PY{o}{=} \PY{l+m+mi}{14}\PY{p}{,} \PY{n}{c} \PY{o}{=} \PY{n}{speed}\PY{p}{)} \PY{c+c1}{\PYZsh{} Define our sensor variable for that c}
            \PY{n}{dist} \PY{o}{=} \PY{n}{sensor}\PY{o}{.}\PY{n}{distance}\PY{p}{(}\PY{p}{)} \PY{c+c1}{\PYZsh{} Take a measurement and assign it to our variable \PYZsq{}dist\PYZsq{}}
            \PY{n+nb}{print}\PY{p}{(}\PY{n}{sensor}\PY{o}{.}\PY{n}{distance}\PY{p}{(}\PY{p}{)}\PY{p}{)} \PY{c+c1}{\PYZsh{} Display the measurement}
            \PY{n}{distance}\PY{o}{.}\PY{n}{append}\PY{p}{(}\PY{n}{dist}\PY{p}{)} \PY{c+c1}{\PYZsh{} Add the measurement to our \PYZsq{}distance\PYZsq{} list}
            \PY{n}{utime}\PY{o}{.}\PY{n}{sleep}\PY{p}{(}\PY{l+m+mi}{1}\PY{p}{)} \PY{c+c1}{\PYZsh{} Wait 1 second before beginning the loop for the next c}
\end{Verbatim}


    When you \texttt{for} loop is complete, ue \texttt{print(distance)} to
see the entire vector of distances collected by the sensor for each
value of \texttt{c}.

We can now start to look at how the distances we measured match up with
the distance that we expected. Using \emph{Canopy}, edit the code below
to plot the values of \texttt{distance} determined by your sensor.

    \begin{Verbatim}[commandchars=\\\{\}]
{\color{incolor}In [{\color{incolor} }]:} \PY{k+kn}{import} \PY{n+nn}{matplotlib}\PY{n+nn}{.}\PY{n+nn}{pyplot} \PY{k}{as} \PY{n+nn}{plt}
        \PY{k+kn}{import} \PY{n+nn}{numpy} \PY{k}{as} \PY{n+nn}{np}
        
        \PY{c+c1}{\PYZsh{} Create an array of the values of c that were used to calculate distance}
        \PY{n}{c} \PY{o}{=} \PY{n}{np}\PY{o}{.}\PY{n}{array}\PY{p}{(}\PY{p}{[}\PY{l+m+mi}{100}\PY{p}{,} \PY{l+m+mi}{150}\PY{p}{,} \PY{l+m+mi}{200}\PY{p}{,} \PY{l+m+mi}{250}\PY{p}{,} \PY{l+m+mi}{300}\PY{p}{,} \PY{l+m+mi}{350}\PY{p}{,} \PY{l+m+mi}{400}\PY{p}{,} \PY{l+m+mi}{450}\PY{p}{]}\PY{p}{)}
        \PY{c+c1}{\PYZsh{} Create an array of the values of \PYZsq{}distance\PYZsq{} measured by your sensor}
        \PY{n}{distance\PYZus{}measured} \PY{o}{=} \PY{n}{np}\PY{o}{.}\PY{n}{array}\PY{p}{(}\PY{p}{[}\PY{p}{]}\PY{p}{)}
        
        \PY{c+c1}{\PYZsh{} Create a plot that shows the relationship between \PYZsq{}c\PYZsq{} and \PYZsq{}distance\PYZsq{}}
        \PY{n}{plt}\PY{o}{.}\PY{n}{plot}\PY{p}{(}\PY{n}{c}\PY{p}{,}\PY{n}{distance\PYZus{}measured}\PY{p}{)}
        \PY{c+c1}{\PYZsh{} Draw a line at the known distance}
        \PY{c+c1}{\PYZsh{} Insert the expected measurement (in mm) as y in the line below}
        \PY{n}{plt}\PY{o}{.}\PY{n}{axhline}\PY{p}{(}\PY{n}{y}\PY{o}{=} \PY{p}{,}\PY{n}{linestyle}\PY{o}{=}\PY{l+s+s1}{\PYZsq{}}\PY{l+s+s1}{\PYZhy{}\PYZhy{}}\PY{l+s+s1}{\PYZsq{}}\PY{p}{)}
        \PY{n}{plt}\PY{o}{.}\PY{n}{xlabel}\PY{p}{(}\PY{l+s+s1}{\PYZsq{}}\PY{l+s+s1}{Speed of Sound (m/s)}\PY{l+s+s1}{\PYZsq{}}\PY{p}{)}
        \PY{n}{plt}\PY{o}{.}\PY{n}{ylabel}\PY{p}{(}\PY{l+s+s1}{\PYZsq{}}\PY{l+s+s1}{Distance From Target (mm)}\PY{l+s+s1}{\PYZsq{}}\PY{p}{)}
        \PY{n}{plt}\PY{o}{.}\PY{n}{show}\PY{p}{(}\PY{p}{)}
\end{Verbatim}


    Based on the intercept of the dashed line (your known distance) and the
solid line (the measured distance at sound speed \emph{c}), create a new
vector of values for \texttt{c} at a finer resolution (intervals of 5?
intervals of 3?) that will help you zoom in on the \texttt{c} value of
that intercept.

Define a new set of values for \texttt{c} on your ESP8266 as you did
above, and determine a new set of vales for \texttt{distance}. Copy and
edit the code you ran in canopy, inserting your new values of \texttt{c}
and \texttt{distance} to create a second figure.

    Try increasing the resolution of \texttt{c} one more time, trying values
with an interval of 1. Again, define a new set of values for \texttt{c}
on your ESP8266 and determine a new set of vales for \texttt{distance}.
Create a third figure with the data.

    Based on the data and figures, what do you estimate the speed of sound
in air to be? What do you notice about the variability in the distance
measurements as you increase the resolution of your \texttt{c} vector?
What would you do to improve accuracy of your measurement?

Connect your ESP8266 to the OTCnet network and submit your estimated
value for speed of sound as a message to the MQTT server in the form of
a string to the \textbf{\oc351\textbackslash{}c\_values} topic by
editing the following code:

    \begin{Verbatim}[commandchars=\\\{\}]
{\color{incolor}In [{\color{incolor} }]:} \PY{k+kn}{from} \PY{n+nn}{simple} \PY{k}{import} \PY{n}{MQTTClient}
        \PY{k+kn}{from} \PY{n+nn}{time} \PY{k}{import} \PY{n}{sleep\PYZus{}ms}
        \PY{n}{server}\PY{o}{=}\PY{l+s+s1}{\PYZsq{}}\PY{l+s+s1}{192.168.89.97}\PY{l+s+s1}{\PYZsq{}}
        \PY{n}{umqtt\PYZus{}client} \PY{o}{=}\PY{l+s+s1}{\PYZsq{}}\PY{l+s+s1}{*Your Name*}\PY{l+s+s1}{\PYZsq{}}
        \PY{err}{`}\PY{n}{c} \PY{o}{=} \PY{n}{MQTTClient}\PY{p}{(}\PY{n}{umqtt\PYZus{}client}\PY{p}{,} \PY{n}{server}\PY{p}{,}\PY{n}{user}\PY{o}{=}\PY{l+s+s1}{\PYZsq{}}\PY{l+s+s1}{*Your Username*}\PY{l+s+s1}{\PYZsq{}}\PY{p}{,}\PY{n}{password}\PY{o}{=}\PY{l+s+s1}{\PYZsq{}}\PY{l+s+s1}{*Your Password*}\PY{l+s+s1}{\PYZsq{}}\PY{p}{)}
        \PY{n}{sleep\PYZus{}ms}\PY{p}{(}\PY{l+m+mi}{2000}\PY{p}{)}      \PY{c+c1}{\PYZsh{} sleep for 2 s}
        \PY{n}{c}\PY{o}{.}\PY{n}{connect}\PY{p}{(}\PY{p}{)}
        \PY{n}{sleep\PYZus{}ms}\PY{p}{(}\PY{l+m+mi}{2000}\PY{p}{)}      \PY{c+c1}{\PYZsh{} sleep for 2 s}
        \PY{n}{pub\PYZus{}topic}\PY{o}{=}\PY{l+s+s1}{\PYZsq{}}\PY{l+s+s1}{ocn351/}\PY{l+s+s1}{\PYZsq{}}\PY{o}{+}\PY{l+s+s1}{\PYZsq{}}\PY{l+s+s1}{c\PYZus{}values}\PY{l+s+s1}{\PYZsq{}}
        \PY{n}{msg}\PY{o}{=}\PY{l+s+s1}{\PYZsq{}}\PY{l+s+s1}{*Your Name*}\PY{l+s+s1}{\PYZsq{}} \PY{o}{+} \PY{n+nb}{str}\PY{p}{(}\PY{o}{*}\PY{n}{Your} \PY{n}{Speed} \PY{n}{of} \PY{n}{Sound} \PY{n}{Value}\PY{o}{*}\PY{p}{)}
        \PY{n}{c}\PY{o}{.}\PY{n}{publish}\PY{p}{(}\PY{n}{pub\PYZus{}topic}\PY{p}{,}\PY{n}{msg}\PY{p}{,}\PY{n}{qos}\PY{o}{=}\PY{l+m+mi}{1}\PY{p}{,}\PY{n}{retain}\PY{o}{=}\PY{k+kc}{True}\PY{p}{)}\PY{err}{`}
\end{Verbatim}


    \hypertarget{part-2-mapping-the-seafloor}{%
\section{Part 2: Mapping the
Seafloor}\label{part-2-mapping-the-seafloor}}

For this activity, you will be using a \textbf{HCSR04} 40 kHz ultrasonic
sensor to determine the bathymetry along the MSB dock. The HCSR04 sensor
sends out a pulse of sound through one transducer, and listens for the
echo return in the second transducer. Based on the delay between sending
the signal and receiving an echo back, you will be able to determine how
far away an object is from the sensor.

This activity uses the waterproof transducer version of the HCSR04
ultrasonic acoustic sensor to create a rough bathymetric chart of a
region of off a pier or dock. For this excercise you will collect
measurements of the depth to the seafloor along the Marine Sciences
Building dock with 1 meter spacing.

\hypertarget{assembling-your-hcsr04-sensor}{%
\subsection{Assembling your HCSR04
Sensor}\label{assembling-your-hcsr04-sensor}}

Begin by connecting the \textbf{HCSR04} to your \textbf{ESP8266}. The
HCSR04 requires 5 volts, so for this excersize you will use your
microcontroller in a breadboard, powered by the USB cable from your
computer The USB cable provides 5V to the board, which means we can use
the \emph{V+} pin on the ESP8266 to give 5V to the HCSR04.

The HCSR04 sensor has 4 pins, \emph{GND}, \emph{VCC}, \emph{Trig}, and
\emph{Echo}. For this excercise, connect the \emph{trig} pin on the
HCSR04 to GPIO pin 12 on the ESP8266, and the \emph{echo} pin to GPIO
pin 14. Connect \emph{GND} to \emph{GND} and \emph{VCC} to \emph{V+}.

The driver for this sensor is already included in the firmware on your
\textbf{ESP8266} as a module named \texttt{hcsr04}. We can define our
sensor using the following code. We will proceed by estimating that the
speed of sound in fresh water is \textbf{1500 meters per second}, so we
will define our variable \texttt{c} as 1500:

\texttt{import\ hcsr04\ sensor\ =\ hcsr04.HCSR04(trigger\_pin\ =\ 12,\ echo\_pin\ =\ 14,\ c\ =\ 1500)}

The \texttt{hcsr04.HCSR04} class requires three input variable: -
\texttt{trigger\_pin}, the GPIO pin on the ESP8266 corresponding to the
\textbf{trig} pin on the HCSR04 - \texttt{echo\_pin}, the GPIO pin on
the ESP8266 corresponding to the \textbf{echo} pin on the HCSR04 -
\texttt{c}, the speed of sound to use to calculate distance

\hypertarget{measuring-distance}{%
\subsection{Measuring Distance}\label{measuring-distance}}

Once your sensor variable is defined, you can take a measurement by
executing the function \texttt{distance}:

\texttt{sensor.distance()}

Your ESP8266 should return an integer value that corresponds to the
distance of the object in front of the sensor in \textbf{millimeters}.
Try pointing your sensor at an object and moving it closer and further
away from it while taking measurements to see how the value changes.

Test the limits of your sensor. Does there appear to be a minimum
distance you can measure? Does there appear to be a maximum distance?
What happens when it seems like you might be too far away or too close
to take a measurement? How accurate do the measurements appear to be
when collected in air?

\hypertarget{bathymetry-measurements}{%
\subsection{Bathymetry Measurements}\label{bathymetry-measurements}}

You will be using the HCSR04 to collect measurements of depth at a
series of fixed positions along the dock. Using the tape measure, start
at the western corner of the dock and take a measurement at 1 meter
spacing. Lower the transducer into the water to the tape mark indicating
10cm is at the surface. Collect and record samples in the following
format:

\begin{longtable}[]{@{}ll@{}}
\toprule
Distance on Dock (m) & Depth (mm)\tabularnewline
\midrule
\endhead
1 & Measurement \#1, \#2, \#3, \#4, \#5\tabularnewline
2 & Measurement \#1, \#2, \#3, \#4, \#5\tabularnewline
\ldots{} & \ldots{}\tabularnewline
\emph{n} & Measurement \#1, \#2, \#3, \#4, \#5\tabularnewline
\bottomrule
\end{longtable}

Collect repeat samples at each fixed distance point. Do all of the
values collected appear to be valid?

Once your sample table is complete, create a plot of the bathymetry in
\emph{Canopy} using the Python code below. For the depth value, take the
median of the measurements at each distance that you believe to be
correct in order to remove bad samples or noise.

    \begin{Verbatim}[commandchars=\\\{\}]
{\color{incolor}In [{\color{incolor} }]:} \PY{k+kn}{import} \PY{n+nn}{matplotlib}\PY{n+nn}{.}\PY{n+nn}{pyplot} \PY{k}{as} \PY{n+nn}{plt} \PY{c+c1}{\PYZsh{} Import the pyplot library}
        \PY{k+kn}{import} \PY{n+nn}{numpy} \PY{k}{as} \PY{n+nn}{np} \PY{c+c1}{\PYZsh{} Import the numpy library}
        
        \PY{c+c1}{\PYZsh{} Create an array of values for the distance along the dock}
        \PY{n}{Distance} \PY{o}{=} \PY{n}{np}\PY{o}{.}\PY{n}{array}\PY{p}{(}\PY{p}{[}\PY{l+m+mi}{1}\PY{p}{,} \PY{l+m+mi}{2}\PY{p}{,} \PY{l+m+mi}{3}\PY{p}{,} \PY{l+m+mi}{4}\PY{p}{,} \PY{l+m+mi}{5}\PY{p}{,} \PY{l+m+mi}{6}\PY{p}{,} \PY{l+m+mi}{7}\PY{p}{,} \PY{l+m+mi}{8}\PY{p}{,} \PY{l+m+mi}{9}\PY{p}{,} \PY{l+m+mi}{10}\PY{p}{]}\PY{p}{)}
        \PY{c+c1}{\PYZsh{} Create an array of values for the distance measured by the sensor (in mm)}
        \PY{n}{Depth} \PY{o}{=} \PY{n}{np}\PY{o}{.}\PY{n}{array}\PY{p}{(}\PY{p}{[}\PY{p}{]}\PY{p}{)}
        \PY{n}{Depth} \PY{o}{=} \PY{n}{Depth}\PY{o}{+}\PY{l+m+mf}{100.} \PY{c+c1}{\PYZsh{} Add 10cm to the measurements to correct for the depth of the transducer}
        \PY{n}{plt}\PY{o}{.}\PY{n}{plot}\PY{p}{(}\PY{n}{Distance}\PY{p}{,}\PY{n}{Depth}\PY{p}{)} \PY{c+c1}{\PYZsh{} Plot the depth at each distance}
        \PY{n}{plt}\PY{o}{.}\PY{n}{ylim}\PY{p}{(}\PY{p}{(}\PY{l+m+mi}{0}\PY{p}{,}\PY{n+nb}{max}\PY{p}{(}\PY{n}{Depth}\PY{p}{)}\PY{p}{)}\PY{p}{)} \PY{c+c1}{\PYZsh{} Set the y\PYZhy{}axis limits}
        \PY{n}{plt}\PY{o}{.}\PY{n}{gca}\PY{p}{(}\PY{p}{)}\PY{o}{.}\PY{n}{invert\PYZus{}yaxis}\PY{p}{(}\PY{p}{)} \PY{c+c1}{\PYZsh{} Invert the y\PYZhy{}axis so the water surface is at the top of the figure}
        \PY{n}{plt}\PY{o}{.}\PY{n}{xlabel}\PY{p}{(}\PY{l+s+s1}{\PYZsq{}}\PY{l+s+s1}{Distance (m)}\PY{l+s+s1}{\PYZsq{}}\PY{p}{)} \PY{c+c1}{\PYZsh{} Label the x\PYZhy{}axis}
        \PY{n}{plt}\PY{o}{.}\PY{n}{ylabel}\PY{p}{(}\PY{l+s+s1}{\PYZsq{}}\PY{l+s+s1}{Depth (mm)}\PY{l+s+s1}{\PYZsq{}}\PY{p}{)} \PY{c+c1}{\PYZsh{} Label the y\PYZhy{}axis}
        \PY{n}{plt}\PY{o}{.}\PY{n}{show}\PY{p}{(}\PY{p}{)} \PY{c+c1}{\PYZsh{} Show the figure}
\end{Verbatim}


    What features, if any, do you see along the bottom?


    % Add a bibliography block to the postdoc
    
    
    
    \end{document}
